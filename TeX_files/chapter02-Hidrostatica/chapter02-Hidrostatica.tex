\chapter{Hidrostática}

\section{Ecuación fundamental de la fluidostática}

\textbf{Fluido en reposo}: No hay esfuerzos tangenciales, y la única
fuerza superficial es la presión.

Equilibrio estático: 

\begin{equation}
\vec{f}_{m}-\vec{\nabla}p=0
\end{equation}


Según el calculo diferencial, 
\[
\vec{\nabla}\times\left(\vec{\nabla}\phi\right)=0\quad\forall\phi\text{ escalar}
\]

\[
\Rightarrow\vec{\nabla}\times\vec{f_{m}}=0.
\]
 $\Rightarrow\vec{f}_{m}$ ha de ser un \emph{campo conservativo}.

\[
\dif p=\vec{f}_{m}\cdot\dif\vec{r}
\]
 Integrando sobre un determinado camino, 
\[
p\left(\vec{r}\right)=p\left(\vec{r}_{0}\right)+\int_{\vec{r}_{0}}^{\vec{r}}\vec{f}_{m}\cdot\dif\vec{r}
\]
 Nos permite calcular la presión en cualquier punto $\vec{r}$ conociendo
el valor en un punto de referencia $\vec{r}_{0}$ y el campo de fuerzas
$\vec{f}_{m}$.

Si $\vec{f}_{m}$ es conservativo 
\[
\vec{f}_{m}=-\rho\vec{\nabla}U
\]
 y, entonces, 
\[
\vec{\nabla}p=-\rho\vec{\nabla}U
\]

Si $\rho$ varia de forma arbitraria, no existen soluciones para la
ecuación , y no es posible llegar al equilibrio, $\rightarrow$ \textcolor{blue}{corrientes
convectivas}

La ecuación sólo admite soluciones cuando $\rho$ es únicamente función
de la presión, o bien es constante (fluido incompresible). 
\[
p+\rho U=cte
\]

$\rightarrow$ \textcolor{blue}{Principio de Pascal}

Hidrostática en el campo de la gravedad

\[
\vec{f}_{m}=\rho\vec{g},
\]
 con 
\[
\vec{g}=-g\vec{k}\qquad\text{donde }g=9.81\,\frac{\textrm{m}}{\textrm{s}^{2}}
\]
 y 
\[
U=gz
\]



Superficies isobáricas (superficies de igual presión), incluida la
superficie libre de los líquidos, horizontales. %

\[
\vec{\nabla}p=-\rho g\vec{k}\Rightarrow\left\{ \begin{aligned}\dparc{p}{x} & =0\\
\dparc{p}{y} & =0\\
\dparc{p}{z} & =-\rho g
\end{aligned}
\right.
\]
%

La presión es únicamente función de la coordenada $z$.

\[
\deriv{p}{z}=-\rho\,g\Rightarrow\,p_{2}-p_{1}
\]

\[
=-\int_{z_{1}}^{z_{2}}\rho\,g\,\dif z
\]
%

\subsection*{Actividad 1:}
\noindent\begin{minipage}[t]{1\columnwidth}%
\begin{itemize}
\item ¿A cuántos metros de columna de agua corresponden la presión atmosférica?
\item Si el aire fuese incompresible, con la densidad que tiene a nivel
del mar, ¿cuál debería ser la altura de la atmósfera para tener la
misma presión?
\end{itemize}
%
\end{minipage}

\section{Presión atmosférica}

La presión atmosférica disminuye con la altura. Dado que el aire es
un gas, su densidad disminuye, en general, cuando disminuye la presión,
por lo que también es menor cuando aumentamos la altura.

Necesitamos información sobre la variación de $\rho$ con $z$, o
bien con $p$.

Opción: aire gas ideal 
\[
\rho=\frac{pM}{RT}\quad\textnormal{con}\,M=28.9\,\textnormal{g/mol}.
\]
 
\begin{equation}
\Rightarrow\,\frac{\dif p}{p}=-\frac{Mg}{RT}\dif z\label{eq:general}
\end{equation}



Sin considerar la variación de $g$ con la altura: 
\begin{itemize}
\item \textcolor{blue}{Atmósfera isoterma:} 
\[
\int_{p_{0}}^{p}\frac{\dif p}{p}=-\int_{0}^{z}\frac{Mg}{RT}\dif z
\]
 
\[
\ln\frac{p}{p_{0}}=-\frac{Mg}{RT}z=-\frac{\rho_{0}g}{p_{0}}z
\]
 
\begin{equation}
\Rightarrow\boxed{p=p_{0}\exp\left(-\frac{\rho_{0}g}{p_{0}}z\right)=p_{0}\exp\left(-\frac{z}{\alpha}\right)}\label{eq:isotermica}
\end{equation}
 donde 
\[
\alpha=\frac{p_{0}}{\rho_{0}g}
\]
\end{itemize}
Valores normales: 
\[
\left.\begin{aligned}\rho_{0} & =1.292\,\text{Kg}/\text{m}^{3}\\
g & =9.80665\,\text{m}/\text{s}^{2}\\
p_{0} & =760\,\text{mmHg}=101328\,\text{Pa}
\end{aligned}
\right\} \rightarrow\alpha=7997.35\,\text{m}\approx8000\,\text{m}
\]



\begin{itemize}
\item \textcolor{blue}{Atmósfera adiabática:} 
\[
\frac{p}{\rho^{\gamma}}=\frac{p_{0}}{\rho_{0}^{\gamma}}\qquad\text{con}\qquad\gamma=\frac{c_{p}}{c_{v}}=1.4\qquad\text{para aire}
\]
\end{itemize}
\[
\dif p=-g\rho\dif z=-\rho_{0}\left(\frac{p}{p_{0}}\right)^{\frac{1}{\gamma}}g\dif z
\]
 
\[
\Rightarrow\frac{\dif p}{p^{\frac{1}{\gamma}}}=-\frac{\rho_{0}}{p_{0}^{\frac{1}{\gamma}}}g\dif z
\]

\[
\int_{p_{0}}^{p}\frac{\dif p}{p^{\frac{1}{\gamma}}}=\int_{0}^{z}-\frac{\rho_{0}}{p_{0}^{\frac{1}{\gamma}}}g\dif z=-\frac{\rho_{0}}{p_{0}^{\frac{1}{\gamma}}}gz
\]


\[
\Rightarrow\frac{1}{-\frac{1}{\gamma}+1}\left.p^{-\frac{1}{\gamma}+1}\right]_{p_{0}}^{p}=-\frac{\rho_{0}}{p_{0}^{\frac{1}{\gamma}}}gz
\]

\[
\Rightarrow\frac{\gamma}{\gamma-1}\left[p^{\frac{\gamma-1}{\gamma}}-p_{0}^{\frac{\gamma-1}{\gamma}}\right]=-\frac{\rho_{0}}{p_{0}^{\frac{1}{\gamma}}}gz
\]

\[
\Rightarrow p^{\frac{\gamma-1}{\gamma}}-p_{0}^{\frac{\gamma-1}{\gamma}}=\frac{1-\gamma}{\gamma}\frac{\rho_{0}}{p_{0}^{\frac{1}{\gamma}}}gz
\]

\begin{equation}
\Rightarrow\boxed{\left(\frac{p}{p_{0}}\right)^{\frac{\gamma-1}{\gamma}}=1+\frac{1-\gamma}{\gamma}\frac{z}{\alpha}}\label{eq:adiabatica}
\end{equation}


\begin{itemize}
\item \textcolor{blue}{Atmósfera estándar:}
\end{itemize}
En realidad, la temperatura media de la atmósfera disminuye de forma
casi lineal con la altura 
\[
T=T_{0}-Bz
\]
 hasta una altura aproximada de 11000 metros (región conocida como
\textit{troposfera}). Los valores de $T_{0}$ (la temperatura a nivel
del mar) y $B$ (\textit{gradiente térmico}) varían no sólo según
el día sino también a lo largo del mismo día. Los valores estándar
usados por convenio son 
\begin{eqnarray*}
T_{0} & = & 15^{\circ}C=288.16\textnormal{K}\\
B & = & 0.0065\textnormal{K/m}
\end{eqnarray*}



\subsection*{Actividad 2:}
Integrar la ecuación (\ref{eq:general}) con esta distribución de
temperatura para obtener 
\begin{equation}
p=p_{0}\left(1-\frac{Bz}{T_{0}}\right)^{\frac{Mg}{RB}}
\end{equation}

El valor del exponente para aire es 
\[
\frac{Mg}{RB}=5.26
\]

Después de la troposfera, la temperatura se mantiene constante hasta
unos 20000 metros para empezar a aumentar de forma gradual.

Hay que tener siempre en cuenta que esta atmósfera estándar es un
valor promediado. 

\section{Fuerza de un fluido estático sobre una superficie}

\subsection{Cálculo de la fuerza}


\begin{center}
\resizebox{0.8\textwidth}{!}{\input{TeX_files/chapter02-Hidrostatica/superficie.pdftex_t}}
\par\end{center}


\[
F=\int_{S}\dif F=\int_{S}(p_{0}+\rho\,g\,h)\dif S==\int_{S}(p_{0}+\rho\,g\,y\,\sin\theta)\dif S
\]

\[
\Rightarrow\;F=p_{0}\,S+\rho\,g\,\sin\theta\int_{S}y\dif S
\]

\begin{description}
\item [{$\int_{S}y\dif S$}] : momento de primer orden de la superficie
$S$ respecto el eje $x$ $\rightarrow$ coordenada $y_{C}$ del centroide
$C$ de la forma
\end{description}
\[
y_{C}\,S=\int_{S}y\dif S\,\Rightarrow\;F=(p_{0}+\rho\,g\,y_{C}\,\sin\theta)S=(p_{0}+\rho\,g\,h_{C})S
\]

\fbox{%
\noindent\parbox[c]{1\textwidth}{%
 La fuerza ejercida sobre una superficie totalmente sumergida se puede
calcular \textbf{imaginando} que la presión que actúa es constante
en toda la superficie e igual al valor en el centroide. %
}}


\subsection{Coordenadas del punto de aplicación}


Momento de la fuerza $\vec{F}$ respecto el eje $x$: 
\[
y_{cp}F=\int_{S}y\dif F=\int_{S}y(p_{0}+\rho\,g\,y\,\sin\theta)\dif S=p_{0}\int_{S}y\dif S+\rho\,g\,\sin\theta\int_{S}y^{2}\dif S
\]
 
\[
\Rightarrow y_{cp}F=\int_{S}y\dif F=p_{0}\,y_{C}\,S+\rho\,g\,\sin\theta I_{xx}
\]
 donde $I_{xx}$ es el momento de segundo orden de la superficie $S$
respecto el eje $x$.

Nuevo sistema de coordenadas $(\xi,\eta,\zeta)$, paralelo a $(x,y,z)$
pero con origen en el centroide $C$. 
\[
I_{xx}=I_{\xi\xi}+y_{C}^{2}\,S\qquad\text{(T. de Steiner)}
\]

\[
\Rightarrow y_{cp}=y_{C}+\frac{I_{\xi\xi}}{\left(y_{C}+\frac{p_{0}}{\rho\,g\,\sin\theta}\right)S}
\]


Para $x_{cp}$: 
\[
\int_{S}x\dif F=\int_{S}x(p_{0}+\rho\,g\,y\,\sin\theta)\dif S=p_{0}\int_{S}x\dif S+\rho\,g\,\sin\theta\int_{S}xy\dif S
\]
 
\[
\Rightarrow\int_{S}x\dif F=p_{0}\,x_{C}\,S+\rho\,g\,\sin\theta I_{xy}
\]
 
\[
I_{xy}=I_{\xi\eta}+x_{C}\,y_{C}\,S\qquad\text{(T. de Steiner)}
\]
 
\[
\Rightarrow x_{cp}=x_{C}+\frac{I_{\xi\eta}}{\left(y_{C}+\frac{p_{0}}{\rho\,g\,\sin\theta}\right)S}
\]



Normalmente, $p_{0}$ (en general, la presión atmosférica) actúa por
igual en las dos caras de la superficie, 
\begin{align*}
F & =\rho\,g\,h_{C}\,S\\
x_{cp} & =x_{C}+\frac{I_{\xi\eta}}{y_{C}\,S}\\
y_{cp} & =y_{C}+\frac{I_{\xi\xi}}{y_{C}\,S}
\end{align*}

Dado que $I_{\xi\xi}$ es, por definición, una cantidad siempre positiva,
el centro de presiones se encuentra siempre por debajo del centroide.


\subsection{Fuerza sobre una superficie curva totalmente sumergida}


\begin{center}
\resizebox{!}{5cm}{\input{TeX_files/chapter02-Hidrostatica/superficie_curva.pdftex_t}}
\par\end{center}

 
\[
\begin{aligned}F_{x} & =-\int_{S}(p_{0}+\rho\,g\,h)\dif S_{x}\\
F_{y} & =-\int_{S}(p_{0}+\rho\,g\,h)\dif S_{y}
\end{aligned}
\]
 

Si proyectamos la superficie $S$ sobre los planos $x=0$ y $y=0$,
obtenemos $S_{x}$ y $S_{y}$, y podemos calcular $F_{x}$ y $F_{y}$,
así como sus puntos de aplicación.

$F_{z}$ resulta ser igual al peso total de fluido que se encuentra
\emph{por encima} de la superficie curva. La linea de acción de $F_{z}$
pasa por el centro de gravedad de la columna de fluido que hay sobre
la superficie.

Las expresiones anteriores son válidas únicamente para fluidos con
densidad constante. Si el fluido está estratificado, de forma que
hay un \textit{gradiente de densidad}, positivo hacia la dirección
vertical negativa, los cálculos se complican.


\subsection*{Actividad 3:}
Calcula la fuerza, y su punto de aplicación, que hace un embalse de agua de 50
metros de profundidad y 200 metros de ancho sobre la pared, vertical,
de la presa.


\section{Principio de Arquímedes}

\fbox{%
	\parbox{1\textwidth}{%
%\begin{quotation}
	\emph{Todo cuerpo sumergido, completa o parcialmente,
		en un fluido experimenta
		un empuje dirigido verticalmente hacia arriba, con magnitud igual al peso del
		fluido desalojado y cuya  linea de acci\'on pasa por el centro de gravedad
		del fluido desalojado}
%\end{quotation} %
}}


\begin{minipage}{0.4\textwidth}
	\begin{center}
		\includegraphics[width=\textwidth]{TeX_files/chapter02-Hidrostatica/arquimedes}
	\end{center}
\end{minipage}
\begin{minipage}{0.5\textwidth}
	Las lineas de acci\'on de las fuerzas de empuje y el peso no
	tienen porqu\'e coincidir, y, en este caso, se producen pares de fuerzas.
	\begin{description}
		\item[\textcolor{blue}{carena}] volumen del fluido desalojado
		\item[\textcolor{blue}{centro
			de carena} o \textcolor{blue}{centro de empuje}] centro de gravedad del fluido desalojado
	\end{description}
\end{minipage}


	El principio de Arquímedes no es, en realidad, un principio. Se puede deducir en cualquier caso simplemente calculando la integral de la presión sobre la superficie que limita el cuerpo.
	
	También se puede obervar que es la resta del peso de la columna de fluido sobre la superficie superior y sobre la superficie inferior.
	
\begin{center}
	\includegraphics[width=0.5\columnwidth]{TeX_files/chapter02-Hidrostatica/arquimedes2}
\end{center}
\section{Segunda ley de Arquímedes}


	La segunda ley de Arqu\'imedes dice que \emph{un cuerpo que flota desaloja su propio peso de fluido}. 
	
	Se puede comprender observando que en la figura, el recipiente con solo fluido y el que tiene fluido y cuerpo flotando, \textbf{deben pesar lo mismo}. Pregunta: ?`C\'omo sabemos que pesan lo mismo?


	\begin{center}
		\includegraphics[width=0.75\columnwidth]{TeX_files/chapter02-Hidrostatica/arquimedes1}
	\end{center}


\section{Estabilidad}

Para un cuerpo sumergido, el centro de gravedad puede ser diferente del centro de empuje, y esto produce un momento que puede ser restaurador (equilibrio estable) o de vuelco (equilibrio inestable)

\begin{center}
	\includegraphics[width=0.7\linewidth]{TeX_files/chapter02-Hidrostatica/estabilidad1}
\end{center}

%\begin{center}
%	\includegraphics[width=0.7\columnwidth]{estab1.png}
%	% estab1.png: 996x491 pixel, 72dpi, 35.14x17.32 cm, bb=
%\end{center}


Para un cuerpo flotante, es m\'as complicado, ya que la posici\'on del centro de empuje var\'ia


Equilibrio estable 
\begin{center}
	\includegraphics[width=0.7\linewidth]{TeX_files/chapter02-Hidrostatica/estabilidad2}
\end{center}



Equilibrio inestable
\begin{center}
	\includegraphics[width=0.7\linewidth]{TeX_files/chapter02-Hidrostatica/estabilidad3}
\end{center}


Pasos para calcular la estabilidad de un cuerpo flotante, consideremos un cuerpo sim\'etrico:

1.- Se calcula posici\'on de equilibrio inicial, mediante las fuerzas $\vec F_E$ y $\vec W$, y sus puntos de aplicaci\'on, $E$ y $G$. Como el cuerpo estan en equilibrio, estas fuerza se alinean con el eje de simetria.

2.- Se realiza una peque\~na perturbaci\'on $\Delta \theta$. El centro de empuje se desplaza a una nueva posici\'on $E'$. La vertical sobre $E'$ corta el eje de simetria en un punto $M$, denominado \textbf{metacentro}. Si el \'angulo $\Delta \theta$ es peque\~no, el metacentro no depender\'a de \'el.

3.- Se calcula la \textbf{altura metac\'entrica}, que es la distancia de $M$ a $G$. Si $M$ est\'a por encima de $G$, la altura metac\'entrica  es positiva, y la posici\'on es \emph{estable}. Si est\'a por debajo, la altura metac\'entrica es negativa, y la posici\'on es \emph{inestable}.

La altua metac\'entrica es una caracter\'istica de la secci\'on transversal del cuerpo y su distribuci\'on de masa.

\begin{center}
	\includegraphics[width=0.7\linewidth]{TeX_files/chapter02-Hidrostatica/estabilidad4}
\end{center}

\begin{center}
	\includegraphics[width=0.7\linewidth]{TeX_files/chapter02-Hidrostatica/estabilidad5}
\end{center}




	La posici\'on del nuevo centro de empuje se calcula con la estimación del centro de masas:

\begin{align*}
	\overline{x}V_{aObdea} &= \int_{Obd}x \dif V - \int_{cOa} x \dif V
\\
	&= \int_{Obd} x L \dif A - \int_{cOa} x L \dif A 
\\
	&= \int_{Obd} x L (x\tan \theta \dif x) - \int_{cOa} x L  (-x\tan \theta \dif x)
\\
	&= \tan \theta \int x^2 2 L \dif x = \tan \theta \int x^2 \dif A 
	\\
	&= \tan \theta I_0
\end{align*}


La altura metac\'entrica es

\begin{equation}
	\overline{MG} = \overline{ME}-\overline{GE}= \frac{\overline{x}}{\tan \theta} - \overline{GE} 
= \frac{I_0}{V_{\textrm{sumergido}}} - \overline{GE} = \frac{\rho g I_0}{W} - \overline{GE}
\end{equation}

Si $\overline{MG}$ es positiva, el equilibrio es estable (para peque\~nas perturbaciones). Si 
$\overline{GE}$ es negativa, el equilibrio es estable siempre